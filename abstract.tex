Dans le domaine de la santé, les nouveau-nés sont des patients requérant des soins très spécifiques, et tout particulièrement en cas de 
naissances prématurées. Ainsi, un monitoring précis de leur débit respiratoire est une aide précieuse de diagnostic pour permettre au 
personnel médical d'effectuer les soins les plus adaptés. Toutefois, les débitmètres actuels sont généralement lents, et peu adaptés aux 
rythmes de respiration rapide des nouveau-nés. \\

L'objectif de ce projet est d'étudier la faisabilité d'un nouveau type de débitmètre respiratoire pédiatrique ultrarapide élaboré par 
nanotechnologie. Il s'appuie sur deux thermocouples nanostructurés synthétisés par croissance électrochimique dans des films minces nanoporeux. 
Les thermocouples encadrent un micro-corps de chauffe : soumis à un flux d'air, la chaleur du corps de chauffe est déportée sur l'un des 
thermocouples et la différence de température subie par les thermocouples indique le débit du flux d'air.  \\

Un banc de test dédié a été réalisé, qui a permis avec succès d'étudier les performances des débitmètres nanostructurés. Le principe de mesure 
de débit a été validé, mais la trop faible puissance du corps de chauffe n'a pas permis une calibration franche du débit. Un micro-corps de 
chauffe dissipant au moins dix fois plus de puissance devra donc être développé. Des mesures réalisées avec une respiration humaine ont montré 
tout le potentiel de cette technologie : d'une part l'extrême sensibilité des thermocouples nanostructurés a permis l'identification claire 
des différents régimes thermiques subis par le débitmètre, et d'autre part des temps de réponse inférieures à 350 ms ont été mesurées, 
démontrant les caractéristiques thermiques uniques de la nanostructuration des thermocouples. Une optimisation des nanostructures est attendue 
pouvoir encore baisser le temps de réponse. \\

Ce travail apporte donc la preuve de faisabilité d'un débitmètre pédiatrique par thermocouples nanostructurés. 



