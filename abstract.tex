% Francais
Dans le domaine de la santé, les nouveau-nés sont des patients requérant des soins très spécifiques, et tout particulièrement en cas de 
naissance prématurée. Un monitoring précis de leurs paramètres vitaux est alors indispensable pour permettre au personnel médical d'effectuer 
les soins les plus adaptés. Parmi ces paramètres le monitoring du débit respiratoire est primordial. Cependant, les très petits volumes 
pulmonaires ainsi que des fréquences respiratoires élevées pousse les appareils conventionnels à leurs limites de détection. L'objectif de ce 
projet est de prouver la faisabilité d'un débitmètre par nanotechnologie. Un tel débitmètre assurerait un temps de réponse absolument prometteur pour 
un prix relativement faible. 

\asterism

La technologie derrière ce débitmètre repose sur l'effet Seebeck, phénomène extrêmement connu dans les thermocouples. Lorsque deux matériaux 
conducteurs ou semi-conducteurs sont associés et soumis à des températures différentes, une tension va être produite entre ces deux matériaux. \\
Le débitmètre respiratoire pédiatrique peut être assimilé à deux thermocouples fonctionnant par effet Seebeck avec un corps de chauffe inclus. 
Quand un flux d'air viendra souffler sur le corps de chauffe, un gradient de température se formera et une tension apparaîtra. Par la très faible 
taille associée aux nanotechnologies, ce type de dispositif assure un temps de réponse extrêmement faible comparé aux appareils actuels. \\

\asterism

Un banc de test a été réalisé afin de pouvoir prouver la faisabilité d'un tel capteur. Ce banc de test permet de tester les échantillons constituant 
le capteur par nanotechnologie avec différents paramètres. 

