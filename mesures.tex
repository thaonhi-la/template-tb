\section{Mesures sur le capteur}
Avant toute chose, afin de pouvoir suivre tous les paramètres et le fonctionnement de notre système, une caméra thermique a été utilisée
pour observer le comportement du corps de chauffe
\begin{comment}
\begin{figure}[H]
    \centering
    \includegraphics[options]{name}
    \caption{Capteur à travers la caméra thermique}
    \label{fig:cameraThermique}
\end{figure}
\end{comment}
Il faut savoir que les pistes d'or réfléchissent énormément. Il est donc difficile d'obtenir un résultat parfait. Une astuce a été de
dessiner au stylo un point sur les pistes à mesurer. Ainsi, les problèmes de réflexions seraient amoindris. La figure \ref*{fig:cameraThermique},
montre que \\

Une seconde mesure permettant de comprendre chaque partie du \gls{capteur} a été effectuée. Celle-ci concerne la résistance du corps de
chauffe.
\begin{table}[H]
    \begin{center}
        \begin{tabular}{|c|c|c|}
            \hline
            Échantillon n\textdegree & Résistance directe [$\Omega$] & Résistance pointes ressort [$\Omega$] \\
            \hline
            D04                      & 28.28                         & 27.58                                 \\
            \hline
        \end{tabular}
        \caption{Résistance à travers les pointes ressort}
        \label{tab:resistancePointeRessort}
    \end{center}
\end{table}
Sur le tableau \ref*{tab:resistancePointeRessort}, une différence entre la resistance mesurée directement sur la piste d'or du corps de
chauffe (résistance appelée Résistance directe) et la résistance à travers les pointes ressort est visible. Ceci peut être dû au fait que
les pointes ressort ont également une petite résistance qui vient alors changer la résistance totale du corps de chauffe. Cependant, il faut
également prendre en compte le fait que la résistance change suivant la position de la mesure. En effet, une mesure faite aux extrémités de
la piste d'or sera différente d'une mesure effectuée sur le centre de la piste d'or.

\section{Dépandances}
\section{Discussion}