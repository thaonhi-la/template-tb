Pour conclure, ce Travail de Bachelor consiste en une première étude sur les débitmètres respiratoires pédiatriques par nanotechnologie. \\
Tout d'abord, il a été prouvé et affirmé que ce type de dispositif reste assurément utile et efficace dans le domaine du diagnostic médical. 
En effet, il permet de suivre la respiration de près et d'avertir diverses maladies telles que les apnées et/ou les détresses respiratoires. \\

De plus, étant donné le caractère prometteur des nanotechnologies sur le temps de réponse ainsi que la sensibilité, une rapidité remarquable 
est attendue par ce type de dispositif. Ces avantages sont particulièrement décisifs chez les prématurés, patients dont les poumons sont encore 
fragiles et donc, sujets à diverses maladies pulmonaires.\\

Un nouveau capteur de débit à alors été développé. Sa réponse à divers environnements a été mesurée et testé afin d'en apprendre d'avantage sur 
ce type de débitmètre. Pour ce faire, un banc de test a été développé et réalisé. Ce banc de test permet de souffler un flux d'air mesurable dans 
le capteur, puis d'obtenir la réponse du dispositif sur un programme LabView (graphe Tension en fonction du temps). \\

Une revue plus précise des points clés du projet se trouve dans les chapitres suivants. 

\section{Revue des objectifs}
\subsection{But principal}
\subsubsection{Prouver la faisabilité d'un débitmètre par nanotechnologie}
\textbf{Accompli}. Les différents résultats obtenus prouvent que le capteur fourni une réponse lorsqu'un flux d'air entre dans le dispositif. Les 
résultats ont montrés qu'une respiration humaine aboutissait à un graphe plus nette et précis qu'une arrivée d'air comprimée. Malgré cela, 
les deux types de flux d'air ont engendré une réponse sortant du bruit général prouvant donc la faisabilité d'un tel capteur !

\subsection{Objectifs spécifiques}
\subsubsection{Elaboration d’un cahier de spécifications \& élaboration d’un catalogue de solutions}
\textbf{Accompli}. L'étude de la littérature a permis un benchmarking des performances à atteindre. Puis plusieurs designs de supports de capteur 
ont été pensé afin de tester les différentes dispositions du capteur. 

\subsubsection{Conception et réalisation d’un débitmètre par nanotechnologie}


\subsubsection{Montage d’un banc de test dédié au débitmètre}
\textbf{Accompli}. Un banc de test constitué d'une arrivée d'air (2 types possibles), d'un débimètre de référence mesurant le débit d'air entrant 
dans le capteur et d'un amplificateur de tension a été réalisé. Malheureusement l'amplificateur de tension n'est pour le moment pas exploitable 
pour cette application. Le reste cependant reste complètement utilisable pour tester le capteur sous différents environnements. 

\subsubsection{Qualification des performances du débitmètre}

\section{Résultats principaux}
\subsection{Délivrabes}
\subsection{Performances}
\section{Perspectives}
\section{Remarques personnelles}
Tout au long du déroulement de ce Travail de Bachelor, j'ai pu mettre en \oe uvre différents aspects étudiés lors de mes études. C'est cet 
aspect concret du projet que j'ai particulièrement apprécié. De plus, j'ai pu partager et m'enrichir avec les connaissances diverses de mes 
collègues. 


\vfil
\hspace{8cm}\makeatletter\@author\makeatother\par
\hspace{8cm}\begin{minipage}{5cm}
    %%if
    % Place pour signature numérique
    \printsignature
    %%fi
\end{minipage}