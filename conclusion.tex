\section{Conclusion générale}
Ce Travail de Bachelor consiste en une première étude sur les débitmètres respiratoires pédiatriques par nanotechnologie. \\
Tout d'abord, l'intérêt de ce type de dispositif n'est plus à démontrer. \\
Il permet d'avertir diverses maladies telles que les apnées et/ou les détresses respiratoires. \\
De plus, étant donné le caractère prometteur des nanotechnologies sur le temps de réponse ainsi que la sensibilité, une rapidité remarquable 
est attendue par ce type de dispositif. Ces avantages sont particulièrement décisifs chez les prématurés, patients dont les poumons sont encore 
fragiles et donc, sujets à diverses maladies pulmonaires.\\
Durant ce travail de Bachelor un nouveau type de capteur de débit a été développé, utilisant des nanofils de matériaux thermoélectriques dans 
une membrane polymère. Un banc de test dédié a été conçu est réalisé, qui permet d'acquérir la réponse thermoélectrique de ces capteurs soumis 
à un flux d'air dans diverses conditions et d'en apprécier les performances. \\

\section{Revue des objectifs}
\subsection*{Objectif général - Prouver la faisabilité d'un débitmètre par nanotechnologie}
\textbf{Accompli}. Les résultats obtenus prouvent que le capteur a une très grande sensibilité au flux d'air. Les 
résultats ont montré qu'une respiration humaine aboutissait à un graphe plus net et précis qu'une arrivée d'air comprimée. Malgré cela, 
les deux types de flux d'air ont engendré une réponse sortant du bruit général prouvant donc la faisabilité d'un tel capteur !

\subsection*{Objectif 1 - Elaboration d'un cahier de spécifications \& élaboration d'un catalogue de solutions}
\textbf{Accompli}. L'étude de la littérature a permis d'établir un benchmarking des performances à atteindre. Puis plusieurs solutions de supports de capteur 
ont été pensé afin de tester les différentes dispositions du capteur. 

\subsection*{Objectif 2 - Conception et réalisation d'un débitmètre par nanotechnologie}
\textbf{Partiellement accompli}. Le capteur \gls{capteur} actuel n'est pas encore un débitmètre. Le principe de mesure de débit par transfert 
de chaleur est validé, mais la très faible puissance du corps de chauffe n'a pas permis de mesurer clairement un débit. En revanche, le principe 
de mesure thermique par deux thermocouples en série a montré son extrême sensibilité, produisant des données uniques sans même l'aide d'un corps 
de chauffe. Le capteur \gls{capteur} nécessite donc encore des améliorations notamment un corps de chauffe plus puissant.


\subsection*{Objectif 3 - Montage d'un banc de test dédié au débitmètre}
\textbf{Accompli}. Un banc de test constitué d'une arrivée d'air (modulable), d'un débitmètre de référence mesurant le débit d'air entrant 
dans le capteur et d'un amplificateur de tension a été réalisé. Il a permis avec succès de tester le capteur sous différents environnements. 

\subsection*{Objectif 4 - Qualification des performances du débitmètre}
\textbf{Partiellement accompli}. L'objectif 2 n'ayant pas permis de produire un débitmètre pleinement fonctionnel, il n'a donc pas été possible 
de qualifier les performances de débitmètre des capteurs \gls{capteur}. Ces derniers ont cependant montré leur grande sensibilité aux flux d'air, notamment 
en termes de rapidité de réaction. Ces performances proposent une mesure de respiration beaucoup plus fine que la simple mesure de débit.  

\section{Résultats principaux}
\begin{itemize}
      \item Une série de capteurs FUN a été réalisé, sur des membranes polycarbonate et polyimide, avec des diamètres de nanofils de tellure de bismuth de
            50, 100 et 200 nm. 
      \item un banc de test a été réalisé, qui permet de mesurer la réponse thermoélectrique des capteurs FUN dans diverses conditions de flux d'air et
            d'alimentation du corps de chauffe, pour les deux soit en mode continu soit en mode pulsé. L'échantillonnage en temps est de 25 ms. 
      \item La mesure de débit par les capteurs FUN s'est révélée infructueuse, le corps de chauffe prévu pour cette fonction manquant notoirement de
            puissance. 
      \item Les capteurs \gls{capteur} montrent une grande sensibilité aux flux d'air en l'absence de corps de chauffe.
      \item Les capteurs \gls{capteur} montrent une sensibilité particulièrement élevée à la respiration humaine.
      \item Les capteurs \gls{capteur} montrent un régime thermique rapide inférieur à 350 ms, attribué à l'échauffement des structures nanofils. Ce régime est
            indépendant du sens du flux. 
      \item Les capteurs \gls{capteur} montrent un régime thermique lent supérieur à 2 s, attribué à la thermalisation de la membrane polymère.
      \item Les résultats mentionnés confirment la faisabilité des structures nanocomposite FUN comme capteurs de respiration pour diagnostique médical.
\end{itemize}

\subsection{Délivrables}
Les délivrables de ce projet sont les suivants :
\begin{itemize}
      \item Rapport technique
      \item Banc de test
      \item Fichiers de conception
\end{itemize}

\newpage
\section{Perspectives}
Plusieurs points ont été étudiés, mais restent à développer. 
\begin{itemize}
      \item Le corps de chauffe n'a pas une forme optimale et son influence reste faible. Il serait alors intéressant de modifier la géométrie de ce
            dernier afin de l'optimiser. Par exemple, une piste d'or en forme de serpentin pourrait améliorer le corps de chauffe.  De plus, une 
            couche d'or plus épaisse permettrait d'y amener un courant plus élevé et ainsi d'avoir un échauffement plus conséquent. 
      \item La réponse en temps de la thermotension néccessiterait un échantillonnage plus fin, de l'ordre de la milliseconde. Une chaîne
            d'acquisition de mesure plus performante devra être développée, notamment en ce qui concerne l'am­plification des petits signaux 
            thermoélectriques.  
      \item Dans le cadre de ce projet, la respiration humaine était réalisée par une personne physique. Une amélioration pourrait être de concevoir
            un système qui simulerait une respiration humaine (avec une certaine humidité et une certaine chaleur). 
      \item Afin d'obtenir un graphe de la réponse du débitmètre de référence ainsi que le graphe du capteur \gls{capteur} correspondant, une
            synchronisation manuelle était réalisée (activer l'oscilloscope en même temps que le programme LabView). Afin d'avoir un temps de réponse très 
            précis, une synchronisation automatique est préférable. 
\end{itemize}

\section{Remarques personnelles}
De manière globale, tout au long du déroulement de ce Travail de Bachelor, j'ai pu mettre en \oe uvre différents aspects étudiés lors de mes études. C'est cet 
aspect concret du projet que j'ai particulièrement apprécié. De plus, j'ai pu partager et m'enrichir avec les connaissances diverses de mes 
collègues. \\
Mais de manière plus ciblée, l'étape de l'état de l'art est une partie que j'ai trouvé passablement longue. En effet, chercher des articles concernant le sujet, les 
décrypter et les trier représente une quantité de travail inattendue. Cependant, c'est une étape intéressante au niveau de 
l'apprentissage des outils de recherche existants et la découverte des technologies innovantes. \\
Les mesures ont également pris passablement de temps, ce qui me semble totalement compréhensible. En revanche, le point que je retiendrai particulièrement 
est l'importance de l'organisation des mesures et des paramètres utilisés. En effet, le nombre de manipulations augmente extrêmement vite et 
il est malheureusement très facile de s'y perdre. \\
Pour finir, ce projet a été fait en parallèle à un cours concernant les dispositifs médicaux. Les informations étudiées lors de ce cours m'ont également 
éclairé sur l'aspect réglementaire obligatoire au lancement d'un dispositif médical tel le débitmètre respiratoire pédiatrique. J'ai ainsi pu 
prendre conscience de la durée non négligeable de la partie concernant les normes et la sécurité ajoutée à la durée du développement du dispositif 
physique. \\
Travailler sur un tel projet a été très motivant, car le domaine du médical est une thématique qui m'intéresse particulièrement 
et dès lors voir les premiers résultats prometteurs du capteur \gls{capteur} est une réussite. 


\vfil
\hspace{8cm}\makeatletter\@author\makeatother\par
\hspace{8cm}\begin{minipage}{5cm}
      %%if
      % Place pour signature numérique
      \printsignature
      %%fi
\end{minipage}