\section{Conclusion générale}
Pour conclure, ce Travail de Bachelor consiste en une première étude sur les débitmètres respiratoires pédiatriques par nanotechnologie. \\
Tout d'abord, il a été prouvé et affirmé que ce type de dispositif reste assurément utile et efficace dans le domaine du diagnostic médical. 
En effet, il permet de suivre la respiration de près et d'avertir diverses maladies telles que les apnées et/ou les détresses respiratoires. \\
De plus, étant donné le caractère prometteur des nanotechnologies sur le temps de réponse ainsi que la sensibilité, une rapidité remarquable 
est attendue par ce type de dispositif. Ces avantages sont particulièrement décisifs chez les prématurés, patients dont les poumons sont encore 
fragiles et donc, sujets à diverses maladies pulmonaires.\\
Un nouveau capteur de débit a alors été développé. Sa réponse à divers environnements a été mesurée et testée afin d'en apprendre davantage sur 
ce type de débitmètre. Pour ce faire, un banc de test a été réalisé. Ce banc de test permet de souffler un flux d'air mesurable dans 
le capteur, puis d'obtenir la réponse du dispositif sur un programme LabView (graphe tension en fonction du temps). \\

Une revue plus précise des points clés du projet se trouve dans les chapitres suivants. 

\section{Revue des objectifs}
\subsection{Prouver la faisabilité d'un débitmètre par nanotechnologie}
\textbf{Accompli}. Les différents résultats obtenus prouvent que le capteur fourni une réponse lorsqu'un flux d'air entre dans le dispositif. Les 
résultats ont montré qu'une respiration humaine aboutissait à un graphe plus net et précis qu'une arrivée d'air comprimée. Malgré cela, 
les deux types de flux d'air ont engendré une réponse sortant du bruit général prouvant donc la faisabilité d'un tel capteur !

\subsection{Elaboration d'un cahier de spécifications \& élaboration d'un catalogue de solutions}
\textbf{Accompli}. L'étude de la littérature a permis d'établir un benchmarking des performances à atteindre. Puis plusieurs solutions de supports de capteur 
ont été pensé afin de tester les différentes dispositions du capteur. 

\subsection{Conception et réalisation d'un débitmètre par nanotechnologie}
\textbf{Non accompli}. Le capteur \gls{capteur} actuel n'est pas encore un débitmètre. En effet, il donne divers résultats, mais reste encore en 
phase de test. Un développement des mesures serait nécessaires afin de comprendre au mieux le comportement du capteur et, ensuite, de lier 
sa réponse à un débit fiable. 


\subsection{Montage d'un banc de test dédié au débitmètre}
\textbf{Accompli}. Un banc de test constitué d'une arrivée d'air (interchangeable), d'un débitmètre de référence mesurant le débit d'air entrant 
dans le capteur et d'un amplificateur de tension a été réalisé. Malheureusement l'amplificateur de tension n'est pour le moment pas exploitable 
pour cette application. Le reste, cependant, reste complètement utilisable pour tester le capteur sous différents environnements. 

\subsection{Qualification des performances du débitmètre}
\textbf{Non accompli}. Comme énoncé plus haut, le débitmètre n'a pas été complètement abouti. Ainsi il n'a pas été possible de qualifier les 
performances de ce dernier. Toutefois, le banc de test qui lui est dédié permet de réaliser les mesures nécessaires à son développement de manière 
efficace. 

\section{Résultats principaux}
Les diverses manipulations permettent de réunir les points clés de ce projet :
\begin{itemize}
    \item La fabrication du capteur est importante et influence le résultat des tests. Une vérification de la résistance du thermocouple permet
          d'anticiper les mesures les moins précises. 
    \item La respiration humaine engendre des résultats bien plus nets que le flux d'air comprimé.
\end{itemize}
Les graphes significatifs de ce projet sont ceux concernant la réponse du capteur à la respiration humaine. \\
En effet, ce sont les résultats les plus concluants puisqu'ils permettent d'observer une réponse claire du capteur \gls{capteur} lorsqu'une 
expiration ou une inspiration a lieu. De plus, le pic de ces deux types de souffle sont opposés, ainsi il est tout à fait possible de les 
différencier. \\
Plus précisément, l'échantillon D06 semble être l'échantillon le plus démonstratif. En effet, les graphes formés à partir de cet échantillon 
montrent plus d'informations qu'avec les autres échantillons, notamment au niveau des différents régimes distinguables. \\
Les mesures ont montré que le temps de réponse d'un tel capteur est inférieur à 500 ms pour tous les échantillons. Cette valeur est positive, mais 
n'est pas encore assez précise. En effet, elle ne permet pas de savoir si ce temps de réponse franchi la limite des 100 ms souhaité ou pas. 

\subsection{Délivrables}
Les délivrables de ce projet sont les suivants :
\begin{itemize}
    \item Rapport technique
    \item Banc de test
    \item Fichiers de conception
\end{itemize}

\section{Perspectives}
Plusieurs points ont été étudiés, mais restent à développer. 
\begin{itemize}
    \item Le corps de chauffe n'a pas une forme optimale et son influence reste faible. Il serait alors intéressant de modifier la géométrie de ce
          dernier afin de l'optimiser. Par exemple, une piste d'or en forme de serpentin pourrait améliorer le corps de chauffe.  De plus, une 
          couche d'or plus épaisse permettrait d'y amener un courant plus élevé et ainsi d'avoir un échauffement plus conséquent. \\
    \item L'amplificateur de tension ne semble pas adéquat pour ce projet. En effet, il amplifie davantage le bruit que le signal à mesurer. Une
          étude plus poussée sur l'électronique et l'amplificateur pourrait être avantageuse. Sans quoi, un autre amplificateur devrait être utilisé. \\
    \item Dans le cadre de ce projet, la respiration humaine était réalisée par une personne physique. Une amélioration pourrait être de concevoir
          un système qui simulerait une respiration humaine (avec une certaine humidité et une certaine chaleur). 
\end{itemize}

\section{Remarques personnelles}
De manière globale, tout au long du déroulement de ce Travail de Bachelor, j'ai pu mettre en \oe uvre différents aspects étudiés lors de mes études. C'est cet 
aspect concret du projet que j'ai particulièrement apprécié. De plus, j'ai pu partager et m'enrichir avec les connaissances diverses de mes 
collègues. \\

Mais de manière plus ciblée, l'étape de l'état de l'art est une partie que j'ai trouvé passablement longue. En effet, chercher des articles concernant le sujet, les 
décrypter et les trier représente une quantité de travail inattendue. Cependant, c'est une étape intéressante au niveau de 
l'apprentissage des outils de recherche existants et la découverte des technologies innovantes. \\

Les mesures ont également pris passablement de temps, ce qui me semble totalement compréhensible. En revanche, le point que je retiendrai particulièrement 
est l'importance de l'organisation des mesures et des paramètres utilisés. En effet, le nombre de manipulations augmente extrêmement vite et 
il est malheureusement très facile de s'y perdre. \\

Pour finir, ce projet a été fait en parallèle à un cours concernant les dispositifs médicaux. Les informations étudiées lors de ce cours m'ont également 
éclairé sur l'aspect réglementaire obligatoire au lancement d'un dispositif médical tel le débitmètre respiratoire pédiatrique. J'ai ainsi pu 
prendre conscience de la durée non négligeable de la partie concernant les normes et la sécurité ajoutée à la durée du développement du dispositif 
physique. \\

Malgré cela, travailler sur un tel projet reste très motivant. En effet, le domaine du médical est une thématique qui m'intéresse particulièrement 
et voir dès lors les premiers résultats prometteurs du capteur \gls{capteur} est une réussite. 


\vfil
\hspace{8cm}\makeatletter\@author\makeatother\par
\hspace{8cm}\begin{minipage}{5cm}
    %%if
    % Place pour signature numérique
    \printsignature
    %%fi
\end{minipage}